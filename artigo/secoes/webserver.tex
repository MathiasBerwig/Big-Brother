\section{Servidor Web}

O servidor web é responsável por atender as solicitações de registro de ponto do Arduino e consulta do Android. Para isso, a tecnologia empregada no desenvolvimento do web server foi o Apache, uma plataforma de servidor gratuita e de código aberta. 

Para atender os requisitos mínimos da aplicação forma necessários apenas três métodos com algumas variações, totalizando cinco rotinas diferentes. A tabela ~\ref{ws_tabela_metodos} demonstra esses métodos e os respectivos retornos.

\begin{table}[H]
\caption{Métodos suportados pelo servidor web.}
\label{ws_tabela_metodos}
\begin{tabularx}{\textwidth}{|c|c|c|X|}
\hline
\textbf{Nome do Método} & \textbf{Requisição Suportada} & \textbf{Parâmetros} & \multicolumn{1}{c|}{\textbf{Retorno}} \\ \hline
registrarPonto & POST & tag & HTTP Code: 201 (sucesso) ou 400 (falha). \\ \hline
getUsuarios & GET & - & Todos os usuários cadastrados. \\ \hline
getRegistros & GET & - & Todos os registros de ponto. \\ \hline
getRegistros & GET & tag & Todos os registros de ponto para a tag informada. \\ \hline
getRegistros & GET & dataInicio, dataFim & Todos os registros de ponto entre a data e hora informados. \\ \hline
\end{tabularx}
\end{table}

Os métodos do tipo \textit{GET} realizam uma consulta no banco de dados, filtrando os resultados de acordo com os parâmetros informados e então retornam as informações em formato \textit{JSON}. O método \textit{registrarPonto} é responsável por realizar uma operação de inserção no banco de dados, e seu retorno é em formato de código HTTP. Caso a inserção tenha sido realizada com sucesso, o mesmo retorna o código 201. Do contrário, 400.