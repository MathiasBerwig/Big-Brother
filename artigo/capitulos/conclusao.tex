\chapter{Conclusão}
\label{conclusao}

A implementação das funcionalidades solicitadas foi cumprida, visto que o sistema é capaz de ler cartões com suporte a RFID, registar as leituras no servidor e exibir ao usuário por meio do aplicativo para Android. O projeto se mostrou uma alternativa simples e barata para o controle de ponto na universidade.

O desenvolvimento apresentado primou pela simplicidade, o que nem sempre é a melhor alternativa para sistemas de utilizados em ambiente de produção. Para tal, seriam necessários diversos ajustes de modo a tornar o sistema mais seguro e resistente à falhas, além de adapta-lo ao sistema utilizado na universidade.

Atividades práticas são muito importantes para o desenvolvimento das habilidades de estudantes, podendo ser o diferencial na experiência profissional dos mesmos. Deste modo, o autor considera que construir protótipos como este é uma tarefa claramente benéfica ao aprendizado dos alunos.